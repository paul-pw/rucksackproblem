\documentclass[12pt, a4paper, ngerman]{article}
\title{Das Rucksackproblem}
\author{Paul Walker und Tom Hofer}
\date{30.06.2022}
\newcommand{\Autor}{Paul Walker und Tom Hofer}
\newcommand{\Kurs}{TINF20IN}
\newcommand{\MatrikelNummer}{3610783, XXXXXX} % TODO(Tom) hier matrikelnummer
\newcommand{\Was}{Hausarbeit Rucksackproblem}
\newcommand{\Studiengang}{Diskrete Mathematik}

\usepackage{biblatex} % für bibliografie
\usepackage{hyperref} % für links zum klicken
\usepackage{color}    % für Farben (benötigt für listings)
\usepackage{listings} % code schnipsel
\usepackage[ngerman]{babel} % lokalisierung der Titel (Inhaltsverzeichniss)
\usepackage{bookmark} % bookmarks für das PDF
\usepackage{csquotes} % korrekte quotes
\usepackage[version=3]{acro} % akronyme
\usepackage{geometry} % seitengeometrie (margin etc einstellen)
\usepackage{parskip}  % zeilenabstand bei neuem paragraph statt indentierung
\usepackage{fancyhdr} % header und footer
\usepackage{array}    % für bessere Tabellen
\usepackage{titlesec} % um die Titel anzupassen
\usepackage{amsfonts} % für \mathbb

\hypersetup{
  pdfauthor={\Autor},
  pdftitle={\Was},
  hidelinks
}

\geometry{
  a4paper,
  left=25mm,
  right=25mm,
  headheight=125mm,
  top=35mm,
  bottom=30mm,
  footskip=15mm
}

% title setup 
% make paragraph have a newline
\titleformat{\paragraph}
{\normalfont\normalsize\bfseries}{\theparagraph}{1em}{}
\titlespacing*{\paragraph}
{0pt}{3.25ex plus 1ex minus .2ex}{1.5ex plus .2ex}

% add bibliography
\addbibresource{bibliography.bib}

% header and footer setup
\pagestyle{fancy}
\fancyhf{}
\rhead{\Was}
\lhead{\leftmark}
\lfoot{Autor: \Autor, Kurs: \Kurs}
\rfoot{Seite \thepage}
\renewcommand{\headrulewidth}{1pt}
\renewcommand{\footrulewidth}{1pt}
\fancypagestyle{simple}{
  \fancyhf{}
  \rhead{\Was}
  \lfoot{Autor: \Autor, Kurs: \Kurs}
  \rfoot{Seite \thepage}
}

% acronyms
\acsetup{
  list/display = used,
  pages/display = first
}

% TODO Acronyms here
%\DeclareAcronym{mvvm}{short=MVVM, long=Model-View-Viewmodel}

\newcommand{\reals}{\ensuremath{\reals}}
\newcommand{\natnums}{\ensuremath{\natnums}}

% code snippet setup
\renewcommand{\lstlistingname}{Code-Auszug}
\renewcommand{\lstlistlistingname}{Liste der Code-Auszüge}

\definecolor{black}{rgb}{0,0,0}
\definecolor{green}{rgb}{0,0.5,0}
\definecolor{orange}{rgb}{1,0.45,0.13}		
\definecolor{brown}{rgb}{0.69,0.31,0.31}

% JavaScript
\lstdefinelanguage{JavaScript}{
    morekeywords={typeof, new, true, false, catch, function, return, null, catch, switch, var, if, in, while, do, else, case, break},
    morecomment=[s]{/*}{*/},
    morecomment=[l]//,
    morestring=[b]",
    morestring=[b]'
}

\lstdefinestyle{light}{
    % General design
    basicstyle={\footnotesize\ttfamily},   
    frame=b,
    % line-numbers
    xleftmargin={0.75cm},
    numbers=left,
    stepnumber=1,
    firstnumber=1,
    numberfirstline=true,	
    % Quellcode design
    identifierstyle=\color{black},
    keywordstyle=\color{blue}\bfseries,
    ndkeywordstyle=\color{green}\bfseries,
    stringstyle=\color{orange}\ttfamily,
    commentstyle=\color{brown}\ttfamily,
    % Quellcode
    language=JavaScript,
    alsodigit={.:;},
    tabsize=2,
    showtabs=false,
    showspaces=false,
    showstringspaces=false,
    extendedchars=true,
    breaklines=true,
}

\begin{document}

% Titlepage
\makeatletter
\begin{titlepage}
  \begin{center}
    \vspace*{1cm}
    {\Huge\scshape \Was}\\[2cm]
    \begin{center}
      \linespread{1}\Huge \@title\\[2cm]
    \end{center}
    {\large \Studiengang}\\
    {\large Dualen Hochschule Baden-Württemberg\\ Stuttgart}\\[2cm]
    {\large von}\\
    {\large\bfseries \@author}
    \vfill
  \end{center}
  \begin{tabular}{l@{\hspace{2cm}}l}
    Matrikelnummer: & \MatrikelNummer \\
    Abgabedatum:    & \@date          \\
  \end{tabular}
\end{titlepage}
\makeatother

% Table of content
\tableofcontents

% TODO entfernen wenn nicht weiter benötigt
\newpage
\thispagestyle{simple}
\printacronyms[name=Abkürzungsverzeichnis, heading=section*]
\newpage

%%%%%%
% Content here
%%%%%% 

\section{Einleitung}

\section{Problemstellung}

Ein Rucksack hat eine bestimmte Tragekapazität (z.b. 5 Kg)
ein Dieb packt bei einem Wohnungsraub sein Diebesgut in diesen Rucksack.
In der Wohnung befinden sich Gegenstände mit unterschiedlichem Gewicht und Geldwert.
Gegenstände können verschieden oft vorhanden sein
(z.b. Schmuck 200g 1000€ 1stk, Elektrogeräte 2kg 500€ 3stk, Kleidung 1kg 100€ 1stk, Geld 300g 200€ 2stk).
Der Dieb möchte bei dem Wohnungsraub einen möglichst großen Gewinn erzielen,
er möchte also den Geldwert der in den Rucksack gepackten Gegenstände maximieren,
da der Rucksack aber nur eine bestimmte Kapazität hat, muss zuerst ein Optimierungsproblem gelöst werden.
Dieses Problem nennt sich das \emph{Rucksackproblem} (oder Englisch: \emph{Knapsack Problem})

Mathematisch formuliert: Es gibt \(m\) Gegenstände.
Sei \(c_i\in\reals : i\in I\) der Wert des Gegenstandes \(i\),
\(a_i\in\reals : i\in I\) das Gewicht des Gegenstandes \(i\)
und \(u_i\in \natnums : i\in I\) wie oft der Gegenstand vorhanden ist jeweils mit \(i\in I = \{1 ,2 ,\ldots ,m\} \).
Die Tragekapazität des Rucksacks ist \(b\in\reals\).
Dann wird für den maximal erreichbaren Geldwert in Abhängigkeit zur Tragekapazität des Rucksacks \(f(b)\) definiert:

\[
  f(b)=\max(\sum_{i=1}^m c_i x_i : \sum_{i=1}^m a_i x_i\leq b, 0\leq x_i\leq u_i,  x_i\in\natnums, i\in I)
\]

Das hier dargestellte Rucksackproblem ist ein begrenztes Rucksackproblem.
Beim unbegrenzen Rucksackproblem kann \(x_i\) jeden beliebigen wert in \(\natnums \) annehmen und ist nicht von \(u_i\) begrenze.

Häufigste Rucksackproblem ist aber das 0-1-Rucksackproblem.
ein Begrenztes Rucksackproblem kann zu einem 0-1-Rucksackproblem vereinfacht werden, indem \(u_i=1\). Dann gilt:

\[
  f(b)=\max(\sum_{i\in I}c_i : \sum_{i\in I}a_i\leq b, i\in I)
\]

Anschaulich bedeutet das, dass jeder Gegenstand nur genau einmal vorhanden ist
und daher auch nur einmal mitgenommen werden kann. % TODO umwandlung normales Rucksack problem zu 01 problem

Das Rucksackproblem gehört zur Kategorie der am schwersten zu lösenden Problemen,
den \(\mathcal{NP}\)-complete Problemen\cite{mainKnapsack}.
In der Praxis gibt es aber einige solide Lösungsverfahren und einige Approximationsverfahren.

\subsection{NP-complete Probleme}

\section{Praktische Anwendungen}

Das Rucksackproblem tritt in der realen Welt häufiger bei Optimierungsproblemen,
speziell bei Problemen in der Ressourcen Allokation auf.
Man denke beispielsweise an einen LKW, der ein bestimmtes Transportvolumen hat und Güter,
die bei verschiedenem Volumen einen unterschiedlichen Gewinn erzielen.
Oder ein Containerschiff das ein bestimmtes Gewicht Tragen kann
und Container mitunterschiedlichem Gewicht und Gewinn.

% TODO remove potentially-->
Eine Problemvariante des Rucksackproblems, die Subset sum wird in der Kryptographie verwendet,
beispielsweise beim Merkle-Hellman-Kryptosystem (das sich nicht als besonders sicher herausstellte)

\section{Lösungsansätze}

viele Lösungsansätze basieren auf dynamische Programmierung

\section{Dynamische Programmierung}


\printbibliography

\end{document}